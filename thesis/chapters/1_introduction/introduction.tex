% -*- root: ../thesis.tex -*-
%!TEX root = ../thesis.tex
% ******************************* Thesis Chapter 1 ****************************

% the code below specifies where the figures are stored
\ifpdf
    \graphicspath{{1_introduction/figures/PNG/}{1_introduction/figures/PDF/}{1_introduction/figures/}}
\else
    \graphicspath{{1_introduction/figures/EPS/}{1_introduction/figures/}}
\fi
% ----------------------------------------------------------------------
%: ----------------------- introduction content ----------------------- 
% ----------------------------------------------------------------------


Machine learning has become a wide field of research with a variety of sub-field, each dedicated to solve various problems in different ways.
One field in particular, usually called \textit{probabilistic machine learning} aims at representing the statistical side of the different models.

\begin{itemize}
    \item Motivate the idea of Bayesian machine learning
    \item Bring to the concept of relation between representation and inference
    \item Introduce each of the chapter properly
\end{itemize}


\section{Bayesian Machine Learning}

\begin{itemize}
    \item Bayes is awesome
    \item Frequentist vs Bayesian
\end{itemize}

Some problems in machine learning have critical requirements, such as probability guarantees for the predictions or working with meaningful interpretable models.
The Bayesian paradigm brings both by inferring probability distributions instead of point estimates.
For a given model, by setting a prior distribution over the parameters, the \textit{posterior} represent the updated belief we have about our model after observing some data.
This allows to model uncertainty in a principled way and to prevent overfitting in the low-data regime.
However, they come at a higher computational cost: a distribution contains more information than a single point, finding analytical solutions is rare and often require involved manual derivations.

% Bayesian Machine Learning is just another name for applied statistics on clean datasets.
A typical example is in medicine, where data is scarce, but the predictive outcome can have a dramatic effect (diagnosis, prognosis, etc...).



\section{The underestimated importance of representation}

\begin{itemize}
    \item Different representation lead to very different results, efficiency etc.
    \item Mention existing approaches
\end{itemize}


\section{The use of Gaussian Processes}

\begin{itemize}
    \item All these things you can do with Gaussian processes
    \item why \ac{GPs} vs other things
\end{itemize}

One of the strong

\section{Thesis Outline}

This thesis is constructed as follows:
\begin{itemize}
    \item Chapter 2 will introduce in details all the common concepts to Bayesian inference and \ac{GPs}.
          This background is generally introduced in each of the published articles, but this chapter allows going more in-depth in the background theory.
          Bayesian inference will be properly introduced with a focus on variational inference and sampling.
    \item Chapter 3 introduces the paper \com{Put paper name here}, which was the first step in this work using augmentations to improve and scale up inference.
    \item Chapter 4 introduced the paper \com{Put paper name here}.
          This paper brings new concepts of augmentation to a much more complex problem: multiclass classification.
    \item Chapter 5 introduces the paper \com{Put paper name here}.
          This work was the first generalization of one type of augmentation and allowed to get a much better understanding of these concepts.
    \item Chapter 6 introduces a completely different way of performing variational inference with Gaussian distribution by using a continuous flows and particles.

\end{itemize}

