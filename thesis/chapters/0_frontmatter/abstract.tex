
% Thesis Abstract -----------------------------------------------------
\ifCLASSINFOlangDE
\selectlanguage{english}
\fi

%\begin{abstractslong}    %uncommenting this line, gives a different abstract heading
\begin{abstracts}        %this creates the heading for the abstract page
\addcontentsline{toc}{chapter}{Abstract}
Performing inference on probabilistic models can represent a challenge even in seemingly simple problems.
When working with non-conjugate Bayesian models, approximate methods such as variational inference or sampling are needed, each with its pitfalls and limits.
For instance, heavy-tailed distributions represent a challenge for sampling methods, and heavily correlated variables quickly become a bottleneck for many inference algorithms.
Instead of developing new state-of-the-art inference algorithms, we focus on reinterpreting models such that standard inference algorithms like blocked Gibbs sampling, usually restricted to more trivial models, become the best choice.
In the first part, we derive model augmentations for different Gaussian Process models such as classification and multi-class classification.
We focus on the effect on inference, and we develop a generalization for a given class of likelihoods.
We show that augmentations are scalable with data and outperform all existing methods in terms of speed and stability.
The second part focuses on a specific approximate approach based on a Gaussian variational distribution.
We show that by representing the Gaussian distribution as a set of particles instead of its parameters, the inference is more flexible, and we prove theoretical convergence bounds.
In addition to the published papers, we discuss the impact of these different augmentations, including their limitations.
We also expose many outlooks on new research directions, including work in progress.
In particular, we present ways to compensate for issues raised in the papers and present new augmentation models and new inference approaches compatible with augmented models.


\end{abstracts}
%\end{abstractlongs}
\ifCLASSINFOlangDE
\selectlanguage{german}
\fi

% ---------------------------------------------------------------------- 
